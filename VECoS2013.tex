 % ewic.tex for classfile V2.04, 6 July 2011

\documentclass{ewic}
%\documentclass[cm]{ewic}
\usepackage{graphicx,wrapfig,subfig}
\begin{document}

\runningheads{Ferrari $\bullet$ Mazzanti $\bullet$ Spagnolo}{A model checker solution for the deadlock free the subway}

\conference{Proceedings of  International Workshop on Verification and Evaluation of Computer and Communication Systems 2013}

\title{A model checker solution for the deadlock free the subway[tp]}

\authorone{Alessio Ferrari\\
ISTI-CNR\\
via G. Moruzzi 1 \\56124 Pisa\\
 ITALY\\
fmt.isti.cnr.it\\
\email{alessio.ferrari@isti.cnr.it}}

\authortwo{Franco Mazzanti\\
ISTI-CNR\\
via G. Moruzzi 1\\
56124 Pisa \\
ITALY\\
fmt.isti.cnr.it\\
\email{franco.mazzanti@isti.cnr.it}}

\authorthree{Giorgio O. Spagnolo\\
ISTI-CNR\\
via G. Moruzzi 1 \\
56124 Pisa\\
 ITALY\\
fmt.isti.cnr.it\\
\email{spagnolo@isti.cnr.it}}


\begin{abstract}
This is an abstract.
\end{abstract}

\keywords{Automatic Train Supervisor, CBTC, Deadlock, UMC, Railway, Subway}

\maketitle

\section{Introduction}

This paper describes  many books
available (see, for example, \cite{Companion,KopkaDaly,Lamport}).

\subsection{eWiC: Information for Authors}
You  instructions about submission, style and preparation of your paper.
[studio preliminare]


\section{Background}
%\input{back}
%Esistono sistemi ferroviari di segnalamento e controllo specifici per le metropolitane. I sistemi metropolitani che presentiamo si chiamano Communicated-based Train Control (CBTC). I sistemi CBTC si basano sulle specifiche internazionali
There are railway signalling and train control systems specific to the metro systems. The metropolitan systems that we present are called Communicated-based Train Control (CBTC). CBTC systems are based on international specifications~\cite{ieee1474} ~\cite{cei2007}.

%Il sistema CBTC è diviso in diversi sotto-sistemi. I seguenti sottosistemi fanno parte del sistema CBTC : Automatic Train Protecion (ATP), Automatic Train Operation (ATO), Interloking (IXL) e Automatic Train Supervisor (ATS).
The CBTC system is divided into several sub-systems. The following subsystems are part of the CBTC system: Automatic Train Protecion (ATP), Automatic Train Operation (ATO), Interlocking (IXL) and Automatic Train Supervisor (ATS).

%L'Automatic Train Protecion conosce la posizione dei treni sulla linea e la loro velocità.  L'ATP calcola la curva di frenatura per garantire una sufficiente separazione dei treni e controlla che il limite di velocità non sia superato.  Il sistema calcola e assegna al treno l'autorizzazione al movimento. La MA è la distanza limite che un treno non può superare.  L'MA viene usata per garantire la separazione dei treni.
 
The Automatic Train Protecion knows the position of trains on railway and their speed. The ATP calculates the braking curve in order to ensure a sufficient separation of trains and checks that the speed limit is not exceeded. The system calculates and assigns to the train the Movement Authority (MA). The MA is the limit distance that cannot be shortened by a running train. The MA is used to ensure separation of trains.
 
 %Automatic Train Operation è un sistema che permette al treno la guida automatica, sostituendo il macchinista.
 
The Automatic Train Operation is a system that allows the train driving automatically, replacing the driver.
 
 %Interlocking gestisce in sicurezza la linea e gli scambi permettendo o negando l'instradamento dei treni in accordo con le regole di sicurezza ferroviaria.

The Interlocking safely manages the railway and switches allowing or denying routing of trains in accordance with the rules of railway safety.
%L'Automatic Train Supervisor  supervisiona l'intera metropolitana, gestendo il traffico dei treni attraverso i limiti di velocità lo scheduling dei treni ed il loro routing.

The Automatic Train Supervisor supervises the entire subway managing train traffic, through, speed limits, scheduling and routing of trains.

%i sistemi di sicurezza ferroviara impedisco ai treni di scontrarsi, ma non prevengono situazioni di stallo.
%è di fondamentale importanza che i sistemi che gestiscono il traffico ferroviario, non causino delle sistuazioni di stallo o deaadlock.
Railway safety systems prevent to trains of collide, but do not prevent deadlock. Therefore it is of fundamental importance that the systems that manage rail traffic, do not cause situations deadlock.

%Il deadlock ferroviario è una condizione in cui 2 o più treni non possono più procedere in sicurezza la loro missione prevista. Perché uno o più treni ne occupano il tragitto in direzione opposta.
The deadlock in the railway is a condition in which two or more trains can no longer safely proceed their planned mission. Because one or more trains will occupy the route in the opposite direction.

%L'ATS quindi dovrà implementare un algoritmo di scheduling e routing che garantisca di non creare situazioni di deadlock. 
%Per capire come impedire che il traffico ferroviario di andare in deadlock, introduciamo i seguenti termini specifici.

The ATS will then implement a routing and scheduling algorithm that ensures do not create deadlock. In order to understand how prevent traffic from rail going to deadlock, we introduce the following specific terms.

%timetable is a list of railway journeys arranged according to the time when they arrivals and departures
%La Timetable è una lista di orari e di binari per ogni treno, che indicano la partenza e l'arrivo del treno in ogni stazione.

The Timetable is a list of times and binaries for each train, which indicate the departure and the arrival of the train at each station.

%Si presuppone che la timetable sia consistente, cioè che applicanto l'orario i treni non creeranno mai situazioni di deadlock. quindi queste posso crearsi solo in caso di ritardi dei treni.
It is assumed that the timetable is consistent, that is, by applying the time the trains do not ever create deadlock situations. so these can be created only in the case of train delays.

%L'Itinerario è un tratto di linea di una stazione. Quando questo va dall'entrata della stazione al binario di stazionamento si chiama itinerario di ingresso, quando va dal binario di stazionamento all'uscita dalla stazione si chiama itinerario di uscita.

The Itinerary is a section of a railway line. When this goes from the entrance to the station to the platform is called ``route of entry'', when it goes from platform to the exit station is called ``route of output''.

%La Missione è il servizio affidato ad un treno specifico, che contiene le informazioni riguardanti la stazione di partenza, le fermate da eseguire, in quali binari il treno deve fermasi, quali itinerai deve eseguire per entrare ed uscire dalle stazioni.

The Mission is the service entrusted to a specific train, which contains information about the station of departure and arrival, scheduled stops, in which platform must compelled to stop, which routes must perform to enter and exit stations.
 

%In ambito metropolitano, al contrario di quanto accade nella ferrovia tradizione, molto spesso i binari in cui si fermano i treni sono definiti in maniera statica. Questo accade perchè i binari della metropolitana sono pensati prendere....

In metropolitan railway, in contrast to what happens in the railroad tradition, very often binaries where trains stop are defined statically
 



 
%\begin{enumerate}
%\item Contesto ferroviario: Metropolitate CBTC
%\item Il sistema CBTC
%\item L'ATS il suo compito -> Routing vs TimeTable
%\item definizione di tabella orario 
%\item definizione di missione [statiche siamo in metro]
%\item sicurezza dei treni 
%\item definizione di itinerario 
%\item definizione di deadlock in un sistema ferroviario
%\end{enumerate}



\subsection{Deadlock Pattern definition}

\begin{figure}[htp]
	\begin{centering}	
	\includegraphics[width=0.45\textwidth, clip]{img/rappresentazione}
	\caption{generic representation of a station}
	\label{fig:rappresent}
	\end{centering}
\end{figure}


\begin{itemize}
\item definizione della rappresentazione usata
\item definizione dei deadlock di base:
\item dritto, cicolare o composizione dei precedenti
\end{itemize}

\begin{figure}[htp]
	\begin{centering}	
	\includegraphics[width=0.43\textwidth, clip]{img/deadlinear}
	\caption{Linear Deadlock Pattern}
	\label{fig:linardeadlock}
	\end{centering}
\end{figure}

\section{Method overview}


%Dati il layout di una metropolitana e la tabella orario dei treni che si vogliono far circolare. Viene definito il modello delle missioni dei singoli treni. Il modello delle missioni unito al modello dello stato della linea formano il modello di scheduling del sistema. 
%Il risultato dell'analisi del model chec ker 

%In questo paragrafo noi descriviamo l'approccio utilizzato per verificare ed eliminare i deadlock in una metropolitana.
%In particolare questa è l'analisi fatta dall'ATS per schedulare i treni in caso di ritardo senza generare deadlock.

In this section we describe the approach used to verify and delete deadlock in a subway.
In particular, this is the analysis performed by the ATS to schedule trains in case of delay without generating deadlocks.

%Il nostro approccio si basa sull'utilizzo iterativo di un model checker per verificare la presenza di deadlock di una metropolitana.

Our approach is based on an iterative model checker to verify the presence of deadlocks of a subway.

\begin{figure}[h!]
	\begin{centering}	
	\includegraphics[width=0.45\textwidth, clip]{img/processo1}
	\caption{Overview of the approach: Part I}
	\label{fig:process1}
	\end{centering}
\end{figure}

%La figura~\ref{fig:process} descrivono graficamente l'approccio usato.
The figures ~\ref{fig:process1} ~\ref{fig:process2} describe the approach used to verify if the deadlock are presented.

%%Nella prima fase, che chiamiamo SLP, vengo preparati i dati per effettuare l'analisi di deadlock della linea.
%I dati di partenza sono  il layout della metropolitana e la tabella orario dei treni che si vogliono far circolare. Questi dati sono usati per definire i modelli delle missioni dei singoli treni. 
The figure~\ref{fig:process1} describe the definition of the scheduling model for the next part of the our method.
The source data, for this part, are the subway  layout and the metro timetable. Both are used to define models of the missions of individual trains into the rail network, see figure~\ref{fig:example} and table ~\ref{tab:timetalbe}.


%Invece il layout della metropolitana viene usato anche per definire lo stato iniziale della linea. Cosi è possibile bloccare alcuni tratti di metropolitana perchè impegnati ad esempio da lavori di manutenzione.

The layout of the subway is also used to define the initial state of the line.This consists of the initial position of the trains and it is also possible lock some parts of the subway because it involved such as maintenance or insert the initial position of train.

%I modelli di missione e lo stato iniziale della metropolitana costituiscono il modello di scheduling della metropolitana.

The models of the mission and the model initial state of the subway they represent the model of scheduling of the subway, see figure~\ref{fig:SchedulingModel}.


\begin{figure}[h!]
	\begin{centering}	
	\includegraphics[width=0.45\textwidth, clip]{img/processo2}
	\caption{Overview of the approach: Part II}
	\label{fig:process2}
	\end{centering}
\end{figure}

The figure~\ref{fig:process2} see second part of our approach to verify the absence of deadlock into model. 
%Il modello di scheduling viene analizzato tramite il model checker UMC[inserire riferimento]
The scheduling model is analyzed using the model checker UMC~\cite{mazzanti2010}.
%UMC permette di specificare come un insime di statecharts UML e di verificare proprità relative all'evoluzione del sistema. 
UMC allows to specify a system as a set of UML Statecharts and verify properties related to the evolution of the system.
[regole di transizione del MC]
%Se risultato del modelchecker è controesempio di deadlock. Si identificano i pattern di deadlock e si inserisco delle nuove regole nelle aree scritiche identificate. Altrimenti potremo affermare che il modello è libero di deadlock.

If the result of the model checker is deadlock counter example we identify patterns of deadlocks and insert the new constraints in the critical areas identified, see section~\ref{sec:back}. Otherwise, we can conclude that the model is free of deadlocks. 
%Quindi tutti i treni sono ingrado di arrivare a destinazione senza generare deadlock.
So all the trains, present into timetable, are able to arrive at destination without generating deadlock.
In some case is not able to calculate a result, because the model is too complex, in the case is  possible to split the layout of subway. This is not always possible, in the following sections, we will explain when it is possible to split the layout.


\section{DeadLock Analysis}
\begin{enumerate}
\item introduzione del caso concreto
\item presentazione layout dell'esempio
\item presentazione di una sez di tab orario
\item presentazione stato linea
\item definizione del modello di missione
\item contro esempio di deadlock generato da umc
\item riconoscimento del tipo di deadlock
\item aggiunta di una nuova regola al modello
\item \ldots ripeto dal passo 6 finch\'{e}
\item GOal: tabella orario libera da deadlock
\end{enumerate}
%\input{example}

\section{Related works}
%\input{releted}
\section{Conclusion}
%\input{conclusion}



\pagebreak
\subsection{Notes}
\begin{enumerate}
\item Please separate multiple author surnames by a `\verb+$\bullet$+' within the
\verb+\runningheads{}{}+ command.

\item The class file is set up to handle up to six authors, i.e., \verb+\authorone{}...\authorsix{}+.

\item Note that the required reference style is Harvard. ewic.cls
uses `natbib.sty' to achieve the desired output so you will need
to choose a natbib compatible .bst that gives Harvard style
output. `chicago.bst' would be a good choice.

\item Try to balance the columns on the final page when your paper is submitted.
\end{enumerate}

That really is all you should need to know to prepare your paper
using ewic.cls.\citep{Mills2003}

You do, of course, have the option to call in any of your
favourite packages for setting maths, graphics, computer listings,
etc.

\textbf{Acknowledgments. }
This work was partially supported by the PAR FAS 2007-2013 (TRACE-IT) project.

%\begin{thebibliography}{9}
%
%\bibitem[Kopka and Daly(2004)]{KopkaDaly}
%Kopka, H. and Daly, P.W.  (2004) \textit{A Guide to \LaTeXe:
%Document Preparation for Beginners and Advanced Users} (4th~edn).
%Addison-Wesley.
%
%\bibitem[Lamport(1994)]{Lamport}
%Lamport L. (1994) \textit{\LaTeX: A Document Preparation System}
%(2nd~edn). Addison-Wesley.
%
%\bibitem[Mittelbach and Goossens(2004)]{Companion}
%Mittelbach, F. and Goossens, M., (2004) \textit{The \LaTeX\
%Companion} (2nd~edn). Addison-Wesley.
%
%\end{thebibliography}

\bibliographystyle{chicago}
\bibliography{bibliography}


\end{document}
